\documentclass[11pt,a4paper]{report}
\usepackage{minted} %must be installed
\usepackage{mathtools} % misc math 
\usepackage{amssymb} % math symbols
\usepackage[utf8]{inputenc}
\usepackage[T1]{fontenc}
\usepackage{enumitem} % for [leftmargin=*]
\usepackage{geometry} % for better controll of margins

\usepackage[romanian]{babel} %romanian names

\usepackage[dvipsnames]{xcolor}
\usepackage{hyperref}
\hypersetup{colorlinks=true, linkcolor=cyan, citecolor=green, filecolor=black, urlcolor=blue}
\newmintinline[code]{cpp}{}


% \usepackage{tikz}
% \usetikzlibrary{arrows}
% \usepackage{hyperref}

% \usepackage{sectsty}
% \chaptertitlefont{\centering\LARGE}
% \sectionfont{\Large}

\geometry{a4paper,left=20mm,right=15mm,top=15mm,bottom=25mm}

%\AtBeginEnviroment{enumerate}{}

%\usepackage{datatool} %for putting files and outputs

%\newcommand{\code}{\}
\newminted[cpp]{cpp} {
  linenos,
  frame=single,
  %breaklines,
  %style=vs,
  fontsize=\large,
  escapeinside=@@
}

\newminted[py]{python} {
  linenos,
  frame=single,
  %breaklines,
  %style=vs,
  fontsize=\large,
  escapeinside=@@
}

\usepackage{xparse}

\DeclareDocumentEnvironment{problema}{O {1} o}{
  \begin{enumerate}[leftmargin=*]
    \addtocounter{enumi}{#1}
    \addtocounter{enumi}{-1}
    \item
}{
  \end{enumerate}
  \IfValueT{#2}{
    \cppcode{#2}
  }
}


\newcommand{\mat}[1]{\mathcal{M}_{#1}(\mathbb{R})}
\newcommand{\matmn}{\mat{m\times{}n}}
\newcommand{\matn}{\mat{n}}

\usepackage{tocloft}
\usepackage{tipa}

\renewcommand{\cftchapleader}{\cftdotfill{\cftdotsep}}
%\hypersetup{colorlinks=true, linkcolor=cyan, citecolor=green, filecolor=black,urlcolor=blue}

\newcommand{\labtitle}[2]{
  \chapter*{#1}
  \addcontentsline{toc}{chapter}{#2}
}
\newcommand{\commonFile}[2]{
  \subsection*{\texttt{#1}}
  \label{#2}
  \addcontentsline{toc}{section}{#1}
  \cppcode{src/#1}
}

\title{Data Structures: How to pass 101}

\begin{document}
\section*{Grafuri (Graphs)}
\subsection*{Reprezentări}
\begin{itemize} \setlength{\itemsep}{-2pt}
  \item Matrice de adiacență: \textipa{/d@:@/} %/dəː/
  \item Liste de adiacență: \code{vector<Node> list[NodeCount];}
  \item Lista muchiilor/arcelor: \code{vector<pair<Node, Node>> list;}
\end{itemize}

\subsection*{Grafuri neorientate (Undirected Graphs)}
\begin{description}\setlength{\itemsep}{-2pt}
  \item Noduri (Vertices); Muchii (Edges)
  \item Graf complet (Complete Graph): toate muchiile adiacente
  \item Graf partial (partial?): ștergem muchii
  \item Subgraf (Subgraph): ștergem noduri
  \item Lanț (Walk / Chain): adiacente 2 cate 2
  \item Lanț elementar (Trail): lanț cu noduri distincte
  \item Ciclu (Cycle): lant cu primu' = ultimu'
  \item Ciclu elementar (elementary?): You get the point
  \item Graf aciclic (Acyclic): natürlich
  \item Conex (Connected): $\forall\,v, w\  \exists\,$lanț
  \item Componenta conexă (Connected component): subgraf maximal (aka tătî bucata)
  \item Lanț eulerian (Eulerian Path/ just `Path'): lanț care vizitează fiecare muchie o singura dată
  \item Ciclu eulerian (Eulerian cycle/circuit): możesz to rozgryźć
  \item Graf eulerian (Eulerian Graph): Graf cu ciclu eulerian
  \item Bipartit (Bipartite): putem împarți nodurile în 2 partiții, în care 2 noduri din aceeași partiție nu sunt adiacente
  \item Lanț hamiltonian (Hamiltonian path): lanț elementar cu toate nodurile
  \item Ciclu hamiltonian (Hamiltonian cycle): duh
\end{description}

\subsection*{Grafuri orientate (Directed Graphs)}
\begin{description}\setlength{\itemsep}{-2pt}
  \item Vârfuri (Vertices); Arce (Edges/arc)
  \item Grad interior: câte intră
  \item Grad exterior: câte ies
  \item Drum (Directed Walk): "lanț"
  \item Graph Turneu (Tour): intre oricare 2 varfuri distincte exista un arc
  \item Slab Conex (Weakly connected): daca-l transformam în neorientat, e conex
  \item Tare Conex (Strongly connected): există drum intre oricare 2 varfuri, dus-intors
  \item Modalitati de parcurgere: Breadth first, Depth first
\end{description}
\subsection*{Grafuri ponderate (Weighted Graphs)}
\subsubsection*{Dijkstra}
\begin{py}
function Dijkstra(Graph, source):
    create vertex set Q
    for each vertex v in Graph:             
        dist[v] = INFINITY                  
        prev[v] = UNDEFINED                 
        add v to Q                      
    dist[source] = 0                        
    
    while Q is not empty:
        u = vertex in Q with min dist[u]    
        remove u from Q 
        for each neighbor v of u:           # only v that are still in Q
            alt = dist[u] + length(u, v)
            if alt < dist[v]:               
                dist[v] = alt 
                prev[v] = u 

    return dist[], prev[]
\end{py}
\subsubsection*{Roy-Floyd}
\begin{py}
let dist be a @$|V| \times |V|$@ array of minimum distances initialized to @$\infty$@
let next be a @$|V| \times |V|$@ array of vertex indices initialized to null

procedure FloydWarshallWithPathReconstruction():
    for each edge (u, v) do
        dist[u][v] = w(u, v) # The weight of the edge (u, v)
        next[u][v] = v
    for each vertex v do
        dist[v][v] = 0
        next[v][v] = v
    for k from 1 to |V| do # standard Floyd-Warshall implementation
        for i from 1 to |V|
            for j from 1 to |V|
                if dist[i][j] > dist[i][k] + dist[k][j] then
                    dist[i][j] = dist[i][k] + dist[k][j]
                    next[i][j] = next[i][k]

procedure Path(u, v)
    if next[u][v] @$=$@ null then
        return []
    path = [u]
    while u @$\neq$@  v
        u = next[u][v]
        path.append(u)
    return path
  \end{py}
\section*{Arbori (Trees)}
\begin{description}\setlength{\itemsep}{-2pt}
  \item Subarbori (Subtree): fii
  \item Nivelul unui nod (Level): distanta pana la radacina
  \item Inaltimea unui nod (Height): cel mai lung lant pana la un nod terminal
  \item Arbore partial al unui graf: graf partial care este arbore
  \item Arbore partial de cost minim (minimum-spanning-tree): avem o functie $f$ care determina costul fiecarei muchii, gasim un arbore partial aî suma costurilor sale sa fie minima.
  \end{description}
  
\subsection*{Reprezentări}
\begin{itemize} \setlength{\itemsep}{-2pt}
  \item like a graph: \textipa{/d@:@/} %/dəː/
  \item Ref descendente: \code{struct Node { int data; Vector<Node*> children; };}
  \item Ref ascendente: \code{int parinte[NodeCount] = {-1, ...};}
\end{itemize}
\subsection*{Kruskal}
\begin{py}
algorithm Kruskal(G) is
A := @$\varnothing$@
for each v @$\in$@ G.V do
    MAKE-SET(v)
for each (u, v) in G.E ordered by weight(u, v), increasing do
    if FIND-SET(u) @$\neq$@ FIND-SET(v) then
       A := A @$\cup$@ {(u, v)}
       UNION(FIND-SET(u), FIND-SET(v))
return A
\end{py}
\subsection*{Prim}
\begin{cpp}
void prim(double mat[sz][sz], ssize_t len) {
    int* s = new int[len] {-1}, j;
    for (int i = 1; i < len; ++i) s[i] = 0;
    for (int k = 1; k < len; ++k) {
        double min = inf;
        for (int i = 0; i < len; ++i)
            if (s[i] != -1 && min > mat[i][s[i]]) {
                min = mat[i][s[i]];
                j = i;
            }
        std::cout << "edge: " << j << "-" << s[j] << "\n";
        for (int i = 0; i < len; ++i)
            if (s[i] != -1 && mat[i][s[i]] > mat[i][j])
                s[i] = j;
        s[j] = -1;
    }
}
\end{cpp}
\subsection*{Arbori binari (Binary Tree)}

\begin{description}\setlength{\itemsep}{-2pt}
\item Arbori binari completi (Perfect binary tree): pe fiecare nivel $s$ are exact $2^s$ noduri
\item Aproape complet (Complete binary tree): pe ultimul nivel lipsesc doar primele din stanga/ ultimele din dreapta

\item Ansamblul Heap binar minim (Min-heap): Arbore aproape complet cu cheia oricarui parinte mai mica sau egala cu a fiului
%TODO: add heap deleet and add
\item Heap sort: stergem si punem intr-un vector

\item Coada de priorități (Priority queue): tl;dr a fancy heap
\end{description}
\subsection*{AVL Trees}
\begin{cpp}
struct Node {
    Key key;
    Node *left, *right;
    int h = 1;
    static int height(const Node* p) { return (p == nullptr) ? 0 : p->h; }
    void updateHeight() { h = std::max(height(left), height(right)) + 1; }
    int balanceFactor() const { return height(left) - height(right); }
    static void rightRotate(Node*& n) {
        Node* t = n->left;
        n->left = t->right;
        t->right = n;
        n->updateHeight();
        t->updateHeight();
        n = t;
    }
    static void leftRotate(Node*& n) { /* like above but swap right and left*/ }
    static void balance(Node*& node) {
        node->updateHeight();
        int factor = node->balanceFactor();
        if (factor > 1) {
            if (node->left->balanceFactor() < 0) leftRotate(node->left);
            rightRotate(node);
        } else if (factor < -1) {
            if (node->right->balanceFactor() > 0) rightRotate(node->right);
            leftRotate(node);
        }
    }
    static void add(Node*& node, const Key& key) {
        if (node == nullptr) { node = new Node(key); return; }
        assert(key != node->key, "Value already in tree");
        if (key < node->key) add(node->left, key);
        else if (key > node->key) add(node->right, key);
        balance(node);
    }

    static void remove(Node*& n, const Key& key) {
        if (n == nullptr) { printf("not found?\n"); return; }
        if (key < n->key) remove(n->left, key);
        else if (key > n->key) remove(n->right, key);
        else { // key == n->key
            if (n->left == nullptr) {
                if (n->right == nullptr) { n = nullptr; return; }
                Node* tmp = n;
                n = n->right;
                tmp->right = nullptr;
                delete tmp;
            } else if (n->right == nullptr) {
                Node* tmp = n;
                n = n->left;
                tmp->left = nullptr;
                delete tmp;
            } else {
                constexpr auto findMax = [](Node*& r, auto& findMax) -> Node*& {
                        if (r->right == nullptr) return r;
                        return findMax(r->right, findMax);
                    };
                Node*& tmp = findMax(n->left, findMax);
                n->key = tmp->key;
                remove(n->left, tmp->key);
            }
        }
        if (n == nullptr) return;
        balance(n);
    }
};
\end{cpp}

\subsection*{Binary heap}
\begin{cpp}
Vector<Val> data;
constexpr static int parent(int i) { return (i - 1) / 2; }
constexpr static int left(int i) { return (2 * i + 1); }
constexpr static int right(int i) { return (2 * i + 2); }
constexpr bool empty() const { return data.empty(); }
void push(const val& v) {
    data.push_back(v);
    for (size_t i = data.size()-1; ;) {
        if (i == 0) break;
        size_t j = parent(i);
        if (data[i] > data[j]) {
            std::swap(data[i], data[j]);
            i = j;
        }
    }
}
val pop() {
    val res = std::move(data[0]);
    data[0] = data.back();
    data.pop_back();
    size_t i = 0;
    size_t sz = data.size();
    while (i < sz) {
        size_t j = left(i);
        if (j < sz) {
            if (j+1 < sz && data[j+1] > data[j]) ++j;
            if (data[i] < data[j]) {
                std::swap(data[i], data[j]);
                i = j;
            } else break;
        } else break;
    }
    return res;
}
\end{cpp}
\subsection*{Tablele de dispersie (Hash tables)}
functii: $h(k) = k \mod m $; \quad \quad
$h(k) = \lfloor m(k \varphi - [k\varphi]) \rfloor,\quad \varphi = \frac{\sqrt{5} - 1}{2}$\\
Or: do it again: $h(k, i) = h_1(k)+i \mod m$; \quad \quad
$h(k, i) = h_1(k)+c_1 i + c_2 i^2 \mod m$
\end{document}
kinda todo:

Neorientate:
 grafuri cu mat, liste de adiacenta, lista muchiilor

Grafuri orientate:
grafuri cu mat, liste de adiacenta, lista arcelor
parcurgere bfs, dfs

Ponderate: implement stuff

Arbori:

Ref descendente, ref ascendente
Arbori partiali de cost minim

Arbori binari:
Inordine...
Modalitati de reprezentare
Binary search trees
AVL trees

Tabele de dispertsie (Hash table):
