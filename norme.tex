%\documentclass[11pt,a4paper]{book}
\documentclass[11pt,a4paper]{article}
\usepackage{minted} %must be installed
\usepackage{mathtools} % misc math 
\usepackage{amssymb} % math symbols
\usepackage[utf8]{inputenc}
\usepackage{enumitem} % for [leftmargin=*]
\usepackage{geometry} % for better controll of margins

\geometry{a4paper,left=30mm,right=20mm,top=20mm,bottom=30mm}

\author{Pavel Andrei}
\date{}
\title{Portofoliu Structuri de date}

\newmintedfile[cppcode]{cpp} {
  linenos,
  frame=single,
  %style=vs,
  fontsize=\large
}


\newenvironment{problema}[1][1]{
  \begin{enumerate}[leftmargin=*]
    \setcounter{enumi}{#1-1}
    \item 
}{
\end{enumerate}
}

\newcommand{\mat}[1]{\mathcal{M}_{#1}(\mathbb{R})}
\newcommand{\matmn}{\mat{m\times{}n}}
\newcommand{\matn}{\mat{n}}

\begin{document}
% \maketitle
\section*{Laborator 1}
  \begin{problema}[16]
    Scrieți funcții pentru implemetarea operațiilor specifice pe matrice de numere reale cu $m$ linii și $n$ coloane:
    suma, diferența și produsul al două matrice, produsul dintre o matrice și un scalar real, transpusa unei matrice, norme matriceale specifice\footnote{
      Dacă $A \in \matmn$, atunci
      $||A||_1 = \max\limits_{1\leq j \leq n} \sum\limits_{i=1}^m |a_{ij}|$,
      $||A||_\infty = \max\limits_{1 \leq i \leq m} \sum\limits_{j=1}^{n}|a_{ij}|$,
      $||A||_{F} = \sqrt{\sum\limits_{i=1}^{m} \sum\limits_{j=1}^{n} a_{ij}^{2} }$.
    },
    citirea de la tastatură a componentelor unei matrice, afișarea componentelor matricei.
    Pentru cazul particular al unei matrice patratice de ordin $n$, să se testeze dacă aceasta satisface criteriul de dominanță pe linii\footnote{
      $A \in \matn$ este strict diagonal dominantă pe linii dacă
      $|a_{ii}| > \sum\limits_{
        \substack{
          j = 1 \\
          j \ne i
        }
      }^{n}
      |a_{ij}|$, pentru orice $i = 1,...,n$.
    } sau pe coloane\footnote{
      $A \in \matn$ este strict diagonal dominantă pe colonane dacă
      $|a_{jj}| > \sum\limits_{
        \substack{
          i = 1 \\
          i \ne j
        }
      }^{n}
      |a_{ij}|$, pentru orice $j = 1,...,n$.
    }.
    Se vor folosi tablouri bidimensionale alocate static.
  
  \end{problema}
  
\cppcode{src/lab1_16_norme_ez.cpp}

\end{document}
