\documentclass[11pt,a4paper]{report}
\usepackage{minted} %must be installed
\usepackage{mathtools} % misc math 
\usepackage{amssymb} % math symbols
\usepackage[utf8]{inputenc}
\usepackage[T1]{fontenc}
\usepackage{enumitem} % for [leftmargin=*]
\usepackage{geometry} % for better controll of margins

\usepackage[romanian]{babel} %romanian names

\usepackage{hyperref}

\usepackage{sectsty}
\chaptertitlefont{\centering\LARGE}
\sectionfont{\Large}

\geometry{a4paper,left=25mm,right=20mm,top=20mm,bottom=30mm}

\author{Pavel Andrei}
\date{}
\title{Portofoliu Structuri de date}

%\AtBeginEnviroment{enumerate}{}

%\usepackage{datatool} %for putting files and outputs

%\newcommand{\code}{\}
\newminted[cpp]{cpp} {
  linenos,
  frame=single,
  %style=vs,
  fontsize=\large,
  escapeinside=@@
}
\newcommand{\cppcode}[1]{\input{cpp_latex/#1}}

\usepackage{xparse}

\DeclareDocumentEnvironment{problema}{O {1} o}{
  \begin{enumerate}[leftmargin=*]
    \addtocounter{enumi}{#1}
    \addtocounter{enumi}{-1}
    \item
}{
  \end{enumerate}
  \IfValueT{#2}{
    \cppcode{#2}
  }
}

%\newcommand{\problemaArg}{} % dummy macro
% \DeclareDocumentEnviroment{problema}{O {1} m}{
  % %\renewcommand{problemaArg}{HI}
  % \begin{enumerate}[leftmargin=*]
    % \addtocounter{enumi}{#1}
    % \addtocounter{enumi}{-1}
    % \item
% }{
% \end{enumerate}
% #2
%}


\newcommand{\mat}[1]{\mathcal{M}_{#1}(\mathbb{R})}
\newcommand{\matmn}{\mat{m\times{}n}}
\newcommand{\matn}{\mat{n}}

\usepackage{tocloft}

\renewcommand{\cftchapleader}{\cftdotfill{\cftdotsep}}
%\hypersetup{colorlinks=true, linkcolor=cyan, citecolor=green, filecolor=black,urlcolor=blue}

\newcommand{\lab}[1]{
  \section*{Laborator #1}
  \addcontentsline{toc}{section}{Laborator #1}
}
\newcommand{\labtitle}[2]{
  \chapter*{#1}
  \addcontentsline{toc}{chapter}{#2}
}
\newcommand{\commonFile}[2]{
  \subsection*{\texttt{#1}}\label{#2}
  \cppcode{src/#1}
}

\begin{document}
\tableofcontents
\pagebreak

\labtitle{Fișiere comume}{Fișiere comune}
\commonFile{utils.h}{utilsh}
\commonFile{matrix.h}{matrixh}
\commonFile{vector.h}{vectorh}

\labtitle{Alocarea dinamică a memoriei.\\ Tipuri specifice.}{Alocarea dinamica a memoriei. Tipuri specifice.}
% \maketitle
\lab{1}
  \begin{problema}[16][src/lab1_16_matrix.cpp]
    Scrieți funcții pentru implemetarea operațiilor specifice pe matrice de numere reale cu $m$ linii și $n$ coloane:
    suma, diferența și produsul al două matrice, produsul dintre o matrice și un scalar real, transpusa unei matrice, norme matriceale specifice\footnote{
      Dacă $A \in \matmn$, atunci
      $||A||_1 = \max\limits_{1\leq j \leq n} \sum\limits_{i=1}^m |a_{ij}|$,
      $||A||_\infty = \max\limits_{1 \leq i \leq m} \sum\limits_{j=1}^{n}|a_{ij}|$,
      $||A||_{F} = \sqrt{\sum\limits_{i=1}^{m} \sum\limits_{j=1}^{n} a_{ij}^{2} }$.
    },
    citirea de la tastatură a componentelor unei matrice, afișarea componentelor matricei.
    Pentru cazul particular al unei matrice patratice de ordin $n$, să se testeze dacă aceasta satisface criteriul de dominanță pe linii\footnote{
      $A \in \matn$ este strict diagonal dominantă pe linii dacă
      $|a_{ii}| > \sum\limits_{
        \substack{
          j = 1 \\
          j \ne i
        }
      }^{n}
      |a_{ij}|$, pentru orice $i = 1,...,n$.
    } sau pe coloane\footnote{
      $A \in \matn$ este strict diagonal dominantă pe colonane dacă
      $|a_{jj}| > \sum\limits_{
        \substack{
          i = 1 \\
          i \ne j
        }
      }^{n}
      |a_{ij}|$, pentru orice $j = 1,...,n$.
    }.
    Se vor folosi tablouri bidimensionale alocate static.
  
  \end{problema}
  %\cppcode{src/lab1_16_matrix.cpp}
\lab{2}
  \begin{problema}[18][src/lab2_18_vector.cpp]
  Scrieți funcții pentru implementarea operațiilor specifice pe vectori din $\mathbb{R}^n$: suma, diferența și produsul scalar al doi vectori,
  produsul dintre un vector și un scalar real, negativarea unui vector, norma euclidiană a unui vector,
  citirea de la tastură a celor $n$ componente ale unui vector, afișarea componentelor vectorului sub forma unui $n$-uplu de elemente.
  Se vor folosi tablouri unidimensionale alocate dinamic.
\end{problema}


\labtitle{Tablouri}{Tablouri}
\lab{3}
%\begin{problema}[6][src/lab3_06_punct.cpp]
  %Definiți în C++ un nou tip de dată corespunzător noțiunilor de punct în plan,
  %tip de dată ce va fi denumit \texttt{Punct2D}.
  %\begin{enumerate}[label=\alph*)]
  %\item Scrieti funcții pentru: calculul distanței dintre două astfel de puncte,
    %citirea de pa tastatură și afișarea unei date de tip \texttt{Punct2D}
  %\item Citiți de la tastatură $n$ date de tip \texttt{Punct2D} și, de asemenea,
    %un punct fix $A(a, b)$. Afișați punctele cele mai apropiate și cele mai
    %depărtate de punctul fix $A$ (alocare dimamică).
    %\item Testați dacă toate cele $n$ puncte citie sunt egal depărtate de punctul fix $A$.
    %\item Eliminați punctele cu abcisa sau ordonata negativă din șirul celor $n$ puncte
      %citite de la tastatură (alocare dinamică).
    %\item Dându-se 3 puncte distincte din plan, verificați dacă punctele sunt
      %sau nu coliniare\footnote{
        %Punctele $A(x_A, y_A),\ B(x_B, y_A),\ C(x_C, y_C)$ sunt coliniare
        %$\iff \left|
          %\begin{matrix}
            %x_A & y_A & 1\\
            %x_B & y_B & 1\\
            %x_C & y_C & 1\\
          %\end{matrix}
        %\right| = 0$
      %}. În cazul în care punctele sunt necoliniare, afișați centrul și
      %raza cercului determinat în mod unic de cele 3 puncte\footnote{
        %Dacă punctele $A(x_A, y_A),\ B(x_B, y_A),\ C(x_C, y_C)$ sunt necoliniare,
        %cercul determinat în mod unic de acestea are raza $R = abc/(4S)$,
        %unde $a, b, c$ sunt lungimile laturilor $\Delta\text{-ului }ABC$,
        %iar $S$ este aria acestuia, iar centrul cercului este $O(x_O, y_O)$,
        %cu $x_O = \left[ (x^2_A+y^2_A)(y_B-y_C) +
          %(x^2_B + y^2_B)(y_C-y_A) +
          %(x^2_C+y^2_C) (y_A-y_B) \right]/D $,
        %$y_O = \left[ (x^2_A+y^2_A)(x_C-x_B) +
          %(x^2_B + y^2_B)(x_A-x_C) +
          %(x^2_C+y^2_C) (x_B-x_A) \right]/D $, și
        %$D = 2[x_A(y_B - y_C) + x_B(y_C-y_A) +x_C (y_A-y_B) ] $.
       % 
      %}.
  %\end{enumerate}
  % \end{problema}

\begin{problema}[7][src/lab3_07_system.cpp]
  Folosind structurile de date \texttt{VECTOR} și \texttt{MATRICE} definite la curs și funcțiile necesare,
  rezolvați următorul sistem algebric liniar cu $n$ ecuații și $n$ necunoscute folosind metoda lui Gauß de eliminare.

  \begin{equation*}
    \left\{
    \begin{aligned}
      2x_1 - x_2 &= 1\\
      -x_1 + 2x_2 - x_3 &= 1\\
      -x_2 + 2x_3 - x_4 &= 1\\
      \cdots\cdots\cdots\cdots &\\
      -x_{n-2} + 2x_{n-1}-x_n &= 1\\
      -x_{n-1} + 2x_n &= 1, \hspace{0.5cm} n \in \mathbb{N}, 2 \leq n \leq 50
    \end{aligned}.
  \right.
  \end{equation*}
\end{problema}



\end{document}
