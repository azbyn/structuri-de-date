\documentclass[11pt,a4paper]{report}
\usepackage{minted} %must be installed
\usepackage{mathtools} % misc math 
\usepackage{amssymb} % math symbols
\usepackage[utf8]{inputenc}
\usepackage{enumitem} % for [leftmargin=*]
\usepackage{geometry} % for better controll of margins

\usepackage[romanian]{babel} %romanian names

\usepackage{hyperref}

\usepackage{sectsty}
\chaptertitlefont{\centering\LARGE}
\sectionfont{\Large}


\geometry{a4paper,left=30mm,right=20mm,top=20mm,bottom=30mm}

\author{Pavel Andrei}
\date{}
\title{Portofoliu Structuri de date}

%\AtBeginEnviroment{enumerate}{}

%\usepackage{datatool} %for putting files and outputs

%\newcommand{\code}{\}
\newmintedfile[cppcode]{cpp} {
  linenos,
  frame=single,
  %style=vs,
  fontsize=\large
}
\usepackage{xparse}

\DeclareDocumentEnvironment{problema}{O {1} o}{
  \begin{enumerate}[leftmargin=*]
    \addtocounter{enumi}{#1}
    \addtocounter{enumi}{-1}
    \item
}{
  \end{enumerate}
  \IfValueT{#2}{
    \cppcode{#2}
  }
}

%\newcommand{\problemaArg}{} % dummy macro
% \DeclareDocumentEnviroment{problema}{O {1} m}{
  % %\renewcommand{problemaArg}{HI}
  % \begin{enumerate}[leftmargin=*]
    % \addtocounter{enumi}{#1}
    % \addtocounter{enumi}{-1}
    % \item
% }{
% \end{enumerate}
% #2
% }

\newcommand{\mat}[1]{\mathcal{M}_{#1}(\mathbb{R})}
\newcommand{\matmn}{\mat{m\times{}n}}
\newcommand{\matn}{\mat{n}}

\usepackage{tocloft}

\renewcommand{\cftchapleader}{\cftdotfill{\cftdotsep}}
%\hypersetup{colorlinks=true, linkcolor=cyan, citecolor=green, filecolor=black,urlcolor=blue}

\newcommand{\lab}[1]{
  \section*{Laborator #1}
  \addcontentsline{toc}{section}{Laborator #1}
}
\newcommand{\labtitle}[2]{
  \chapter*{#1}
  \addcontentsline{toc}{chapter}{#2}
}

\begin{document}
\tableofcontents
\pagebreak
  \labtitle{Alocarea dinamică a memoriei.\\ Tipuri specifice.}{Alocarea dinamica a memoriei. Tipuri specifice.}
% \maketitle
\lab{1}
  \begin{problema}[16][src/lab1_16_matrix.cpp]
    Scrieți funcții pentru implemetarea operațiilor specifice pe matrice de numere reale cu $m$ linii și $n$ coloane:
    suma, diferența și produsul al două matrice, produsul dintre o matrice și un scalar real, transpusa unei matrice, norme matriceale specifice\footnote{
      Dacă $A \in \matmn$, atunci
      $||A||_1 = \max\limits_{1\leq j \leq n} \sum\limits_{i=1}^m |a_{ij}|$,
      $||A||_\infty = \max\limits_{1 \leq i \leq m} \sum\limits_{j=1}^{n}|a_{ij}|$,
      $||A||_{F} = \sqrt{\sum\limits_{i=1}^{m} \sum\limits_{j=1}^{n} a_{ij}^{2} }$.
    },
    citirea de la tastatură a componentelor unei matrice, afișarea componentelor matricei.
    Pentru cazul particular al unei matrice patratice de ordin $n$, să se testeze dacă aceasta satisface criteriul de dominanță pe linii\footnote{
      $A \in \matn$ este strict diagonal dominantă pe linii dacă
      $|a_{ii}| > \sum\limits_{
        \substack{
          j = 1 \\
          j \ne i
        }
      }^{n}
      |a_{ij}|$, pentru orice $i = 1,...,n$.
    } sau pe coloane\footnote{
      $A \in \matn$ este strict diagonal dominantă pe colonane dacă
      $|a_{jj}| > \sum\limits_{
        \substack{
          i = 1 \\
          i \ne j
        }
      }^{n}
      |a_{ij}|$, pentru orice $j = 1,...,n$.
    }.
    Se vor folosi tablouri bidimensionale alocate static.
  
  \end{problema}
  %\cppcode{src/lab1_16_matrix.cpp}
\lab{2}
  \begin{problema}[18][src/lab2_18_vector.cpp]
  Scrieți funcții pentru implementarea operațiilor specifice pe vectori din $\mathbb{R}^n$: suma, diferența și produsul scalar al doi vectori,
  produsul dintre un vector și un scalar real, negativarea unui vector, norma euclidiană a unui vector,
  citirea de la tastură a celor $n$ componente ale unui vector, afișarea componentelor vectorului sub forma unui $n$-uplu de elemente.
  Se vor folosi tablouri unidimensionale alocate dinamic.
  \end{problema}
  

\end{document}
